\documentclass[a4paper,10pt]{article}
\usepackage{mypreamble}

%% Page setup
\geometry{
    margin=2cm,
    includehead,
    includefoot,
    headsep=8pt,
    footskip=16pt,
}
\pagestyle{fancy}
\MakeSingleHeader% {<l>}{<r>}
    {\TextCheatsheetEng: Automata Theory}%
    {\TextDiscreteMathEng, \IconSpring~Spring 2024}
\fancyfoot{}
\fancyfoot[L]{\tiny Build time: \DTMnow}
\fancyfoot[R]{\tiny Source code can be found at \url{https://github.com/Lipen/discrete-math-course} \\ made by discrete math lovers \Cat}
% \fancyfoot[C]{\thepage\ of \zpageref{LastPage}}

%% Add custom setup below

% \titlespacing{\type}{left}{above}{below}[right]
\titlespacing{\section}{0pt}{*1}{*0.5}
\titlespacing{\subsection}{0pt}{*1}{*0.5}

\declaretheoremstyle[
    spaceabove=6pt,
    spacebelow=6pt,
    postheadspace=0.5em,
    notefont=\normalfont\scshape,
]{mystyle}
\declaretheorem[style=mystyle]{theorem}

\usetikzlibrary{automata,shapes}
\tikzstyle{myautomatonstyle}=[
    auto, on grid,
    >={Stealth[]},
    shorten >=1pt,
    semithick,
    bend angle=15,
    initial text={},
    every state/.style={
        inner sep=0pt,
        minimum size=2em,
    },
]

\begin{document}

\selectlanguage{english}

\setcounter{section}{5}% +1 = actual
\section{Automata Theory Cheatsheet}

\begin{terms}
    \item \textbf{Alphabet}\Href{https://en.wikipedia.org/wiki/Alphabet_(formal_languages)} is a finite set of symbols, commonly denoted $\Sigma$.

    \item \textbf{Word} $w \in \Sigma^*$ is a finite sequence of symbols from alphabet $\Sigma$
    For example, $w = abacaba \in \Set{a, b, c}^*$.

    \item \textbf{Length} of a word: $|w| = n$, where $n$ is the number of symbols in word $w$.
    For example, $|abacaba| = 7$.

    \item \textbf{Empty word} $\varepsilon$ is a word of length 0.

    \item \textbf{Concatenation} of words $w_1$ and $w_2$ is $w_1 \cdot w_2 = w_1 w_2$.

    \item \textbf{Power} of a word $w$ is $w^n = w \cdot w \cdot \ldots \cdot w$ ($n$ times).
    % Note that $\Sigma^0 = \Set{\varepsilon} \neq \emptyset$.

    \item \textbf{Reverse} of a word $w$ is $w^R$.

    % \item \textbf{Subword} of a word $w$ is $w[i:j] = w_i \dots w_j$.

    % \item \textbf{Prefix} of a word $w$ is $w[1:i] = w_1 \dots w_i$.

    % \item \textbf{Suffix} of a word $w$ is $w[i:n] = w_i \dots w_n$.

    \item \textbf{Language}\Href{https://en.wikipedia.org/wiki/Formal_language} $L$ over an alphabet $\Sigma$ is a set of words $L \subseteq \Sigma^*$.

    \item \textbf{Empty language} $\emptyset$ is a language that contains no words.

    \item \textbf{Singleton\Href{https://en.wikipedia.org/wiki/Singleton_(mathematics)} language} $\Set{w}$ is a language that contains only one word $w$.

    \item \textbf{Empty string language} $\Set{\varepsilon}$ is a language that contains only one empty word $\varepsilon$.

    \item Operation on languages:

    \begin{terms}
        \item \textbf{Union}: $L_1 \union L_2 = \Set{w \given w \in L_1 \lor w \in L_2}$

        \item \textbf{Intersection}: $L_1 \intersection L_2 = \Set{w \given w \in L_1 \land w \in L_2}$

        \item \textbf{Complement}: $\neg{L} = \Set{w \given w \notin L}$

        \item \textbf{Concatenation}\Href{https://en.wikipedia.org/wiki/Concatenation}: $L_1 \cdot L_2 = \Set{ab \given a \in L_1, b \in L_2}$

        \item \textbf{Kleene star (Kleene closure)}\Href{https://en.wikipedia.org/wiki/Kleene_star}: $L^* = \bigunion_{k = 0}^{\infty}\Sigma^k$
    \end{terms}

    % TODO: move
    \item \textbf{Equivalence}: $L_1 \equiv L_2 \iff (L_1 \intersection \overline{L_2}) \union (\overline{L_1} \intersection L_2) = \emptyset$

    % TODO: rewrite
    % \item $\mathrm{REG}$ \textbf{(family of regular languages)}\Href{https://en.wikipedia.org/wiki/Regular_language} is set over an alphabet $\Sigma$.

    \item \textbf{Regular language}\Href{https://en.wikipedia.org/wiki/Regular_language} is a language that can be defined by a regular expression.

    Regular languages are defined inductively (recursively):

    \begin{terms}
        \item The empty language $\emptyset$ is regular.

        \item For any $a \in \Sigma$, the singleton language $\Set{a}$ is regular.

        \item If $A$ is a regular language, then $A^*$ (Kleene star) is also regular.

        \item If $A$ and $B$ are regular languages, then $A \union B$ (union) is also regular.

        \item If $A$ and $B$ are regular languages, then $A \cdot B$ (concatenation) is also regular.

        \item No other languages over $\Sigma$ are regular.
    \end{terms}

    % TODO: remove/rewrite
    \item \textbf{REG (set of regular languages)} is set over an alphabet $\Sigma$\\ $\mathrm{REG} = \bigunion_{k = 0}^{\infty} \mathrm{Reg}_{k} = \mathrm{Reg}_{\infty}$.

    % TODO: remove/rewrite
    \begin{terms}
        \item $\mathrm{Reg}_{0} = \Set{\emptyset, \Set{\varepsilon}} \union \Set{\Set{c} \given c \in \Sigma}$.

        \item $\mathrm{Reg}_{i + 1} = \mathrm{Reg}_{i} \union \Set{A \cdot B, A \union B \given A,B \in \mathrm{Reg}_{i}} \union \Set{A^* \given A \in \mathrm{Reg}_{i}}$.
    \end{terms}

    \item $\mathrm{REG}$ is closed under union, concatenation, and Kleene star operations.

    \item \textbf{Regular expressions (regex)}\Href{https://en.wikipedia.org/wiki/Regular_expression} consist of constants, which denote sets of strings, and operator symbols, which denote operations over these sets.

    % TODO: fix A=\alpha -- this is incorrect
    Hereinafter: $c \in \Sigma$, $A \subseteq \Sigma^*$, $A = \alpha$,  $B \subseteq \Sigma^*$, $B = \beta$.

    % TODO: move table to the right (along the text)
    % TODO: swap columns
    \begin{tabular}{cc}
        \toprule
        \thead{Language} & \thead{Regex} \\
        \midrule
        $\emptyset$ & $\emptyset$ \\
        $\Set{\varepsilon}$ & $\varepsilon$ \\
        $\Set{c}$ & $c$ \\
        $A \union B$ & $\alpha | \beta$ \\
        $A \cdot B$ & $\alpha \beta$ \\
        $A^*$ & $\alpha^*$ \\
        $A\cdot A^*$ & $\alpha^+$ \\
        $A \union \Set{\varepsilon}$ & $\alpha?$ \\
        \bottomrule
    \end{tabular}

    \item \textbf{Deterministic Finite Automaton (DFA)}\Href{https://en.wikipedia.org/wiki/Deterministic_finite_automaton} is 5-tuple $\mathcal{A} = (\Sigma, Q, q_0, F, \delta)$, where:

    \begin{terms}
        \item $\Sigma$ is an alphabet;
        \item $Q = \Set{q_1, \dots, q_n}$ is a finite set of states;
        \item $q_0 \in Q$ is an initial state;
        \item $F \subseteq Q$ is set of final (terminal, accepting) states;
        \item $\delta \colon Q \times \Sigma \to Q$ is transition function.
    \end{terms}

    \item Language accepted by an automaton $\mathcal{A}$ is the set $L(\mathcal{A}) = \Set{w \given \delta(q_0, w) \in F}$.

    \item \textbf{Nondeterministic Finite Automaton (NFA)}\Href{https://en.wikipedia.org/wiki/Nondeterministic_finite_automaton} is 5-tuple $\mathcal{A} = (\Sigma, Q, q_0, F, \delta)$, where:

    \begin{terms}
        \item $\Sigma$ is an alphabet;

        \item $Q = \Set{q_1, \dots, q_n}$ is a finite set of states;

        \item $q_0 \in Q$ is an initial state;

        \item $F \subseteq Q$ is set of final (terminal, accepting) states;

        \item $\delta: Q \times \Sigma \to 2^Q$ is a transition function.

        % TODO: ?
        % $\delta: (q, c) \mapsto \Set{q^_1, q_2, \dots, q_n}$, $c \in \Sigma$, $q \in Q$ (Nondeterminism).
    \end{terms}

    % TODO: explain \vdash first
    % \item Language accepted by an NFA $\mathcal{A}$ is the set $L(\mathcal{A}) = \Set{w \given \Pair{q_0, w} \vdash^*_{\text{NFA}} \Pair{f, \epsilon}, f \in F}$.

    \item \textbf{NFA to DFA} conversion algorithm:

    \begin{enumerate}
        \item Set initial state of NFA to initial state of DFA.

        \item Take the states in the DFA, and compute in the NFA what the union $\delta$ of those states for each letter in the alphabet and add them as new states in the DFA.

        \item Set every DFA state as accepting if it contains an accepting state from the NFA
    \end{enumerate}

    \item \textbf{Epsilon-NFA ($\varepsilon$-NFA)} is an NFA which allows $\varepsilon$-moves, that is, the automaton can change state without consuming input.

    % TODO: inline function definition into the definition above.
    \begin{terms}
        \item $\delta \colon Q \times (\Sigma \union \{\varepsilon\}) \to 2^Q$.
    \end{terms}

    \item \textbf{$\varepsilon$-NFA to NFA}:
    % TODO: prove that each step is correct (does not change the semantics of the automaton, i.e. the language it accepts is the same after each step, including the last one)
    \begin{enumerate}
        \item Find transitive-closure of $\varepsilon$.

        \item Back-propagate accepting states over $\varepsilon$-transitions.

        \item Perform symbol-transition back-closure over $\varepsilon$-transitions.

        \item Remove $\varepsilon$-transitions.
    \end{enumerate}

    % TODO: rewrite
    % TODO: theorem environment
    \item \textbf{Pumping lemma}\Href{https://en.wikipedia.org/wiki/Pumping_lemma_for_regular_languages} states that if $L$ if a regular language, then there exists an integer $n > 1$ depending only on $L$, such that $\forall w \in L$, $|w| > n$ can be written as $w = xyz$, such that:

    \begin{enumerate}
        \item $|y| > 0$, i.e. $y \neq \varepsilon$.

        \item $|xy| \leq n$.

        \item $\forall k \geq 0$, word $x y^{k} z$ is also in language $L$.
    \end{enumerate}


    \item \textbf{Mealy\footnote{\textsc{Mealy, George H.} (1955). A Method for Synthesizing Sequential Circuits.
    \textit{The Bell System Technical Journal}, 34(5), 1045--79.} machine}\Href{https://en.wikipedia.org/wiki/Mealy_machine} is a finite-state machine whose output is determined both by the current state and the current input.

    \begin{minipage}{\linewidth}
    \begin{wrapfigure}{r}{0pt}
        \setlength{\tabcolsep}{2pt}%

        \tikz[myautomatonstyle,baseline=(s0.base)]{
            \node[state, initial] (si) {$s_i$};
            \node[state, above right=1cm and 2cm of si] (s0) {$s_0$};
            \node[state, below right=1cm and 2cm of si] (s1) {$s_1$};
            \path[->]
            (si) edge [bend left, sloped, above] node {0/0} (s0)
            edge [bend right, sloped, below] node {1/0} (s1)
            (s0) edge [loop right, right] node {0/0} (s0)
            edge [bend left, right] node {1/1} (s1)
            (s1) edge [loop right, right] node {1/0} (s1)
            edge [bend left, left] node {0/1} (s0)
            ;
            \node (label) at (5,0) {This Mealy \\ machine's output is 1 \\ if current and \\ previous symbols are \\ equal, else 0};
        }
        \vspace{-\intextsep}
    \end{wrapfigure}

    Formally, $\mathcal{M}_\text{Mealy} = \Set{\Sigma, \Omega, Q, q_0, \delta, \lambda_\text{Mealy}}$, where:

    \begin{terms}
        \item $\Sigma$ is an input alphabet;

        \item $\Omega$ is an output alphabet;

        \item $Q = \Set{q_1, \dots, q_n}$ is finite set of states;

        \item $q_0 \in Q$ is an initial state;

        \item $\delta \colon Q \times \Sigma \to Q$ is a transition function;

        \item $\lambda_\text{Mealy} \colon Q \times \Sigma \to \Omega$ is an output function.
    \end{terms}

    \end{minipage}

    \item \textbf{Moore machine}\Href{https://en.wikipedia.org/wiki/Moore_machine} is a finite-state machine whose output is determines only by the current state.

    \begin{minipage}{\linewidth}
    \begin{wrapfigure}{r}{0pt}
        \setlength{\tabcolsep}{2pt}%

        \tikz[myautomatonstyle]{
            \node[state, initial] (s0) {$0$};
            \node[state, right=2cm of s0] (s1) {$1$};
            \node[state, right=2cm of s1] (s2) {$2$};
            \path[->]
            (s0) edge [loop above, above] node {0} (s0)
            edge [bend left, above] node {1} (s1)
            (s1) edge [bend left, below] node {1} (s0)
            edge [bend right, below] node {0} (s2)
            (s2) edge [bend right, above] node {1} (s1)
            edge [loop above, above] node {0} (s2)
            ;
            \node (label) at (2,-1) {This Moore machine's output \\ is modulo 3 of a binary number};
        }
        \vspace{-\intextsep}
    \end{wrapfigure}

    Formally, $\mathcal{M}_\text{Moore} = (\Sigma, \Omega, Q, q_0, \delta, \lambda_\text{Moore})$, where:

    \begin{terms}
        \item $\Sigma$ is an input alphabet;

        \item $\Omega$ is an output alphabet;

        \item $Q = \Set{q_1, \dots, q_n}$ is a finite set of states;

        \item $q_0 \in Q$ is an initial state;

        \item $\delta \colon Q \times \Sigma \to Q$ is a transition function;

        \item $\lambda_\text{Moore} \colon Q \to \Omega$ is an output function.
    \end{terms}

    \end{minipage}

    % TODO: what is M?
    % TODO: theorem
    \item \textbf{Emptiness}.
    Language $L(M)$ is not empty ($L \neq \emptyset$) if $M$ accepts a word $w$ such that $|w| \leq n$.

    % TODO: what is M?
    % TODO: theorem
    \item \textbf{Infiniteness}.
    Language $L(M)$ is infinite $(|L| = \infty)$ if $M$ accepts a word $w$ such that $n \leq |w| < 2n$.

    % TODO: theorem environment
    \item \textbf{Myhill-Nerode theorem}\Href{https://en.wikipedia.org/wiki/Myhill–Nerode_theorem} states that the following three statement are equivalent:

    \begin{enumerate}
        \item $L \subseteq \Sigma^*$ is accepted by some finite automaton ($L$ is regular).

        \item $L$ is the union of some equivalence classes of right invariant equivalence relation of finite index.

        \item Let $R_L$ be a relation over words: $x \rel[R_L] y$ iff $\forall z \in \Sigma : xz \in L \equiv yz \in L$.
        % TODO: check who exactly is "of finite index" - relation or quiotient.
        Then $\quotient[R_L]{\Sigma^*}$ is of finite index.
    \end{enumerate}

\newpage

    \item \textbf{Formal grammar}\Href{https://en.wikipedia.org/wiki/Formal_grammar} is 4-tuple $\mathcal{G} = (V, T, S, \mathcal{P})$, where:

    \begin{terms}
        \item $\mathcal{V}$ is vocabulary, set of variables or non-terminal symbols.

        \item $T$ is set of terminal symbols disjoint from $\mathcal{V}$.

        \item $S$ is start symbol, also called sentence symbol.

        \item $\mathcal{P}$ is set of production rules, each rule of the form:
        $\mathcal{V}^*S\mathcal{V}^* \xrightarrow{} \mathcal{V}^*$.
    \end{terms}

    \item Binary relation $\mathbf{\Rightarrow}$ over an grammar $\mathcal{G}$ is defined by:

    $x \Rightarrow y \Longleftrightarrow \exists u,v,p,q \in \mathcal{V}: (x = upv) \land (p \rightarrow{} q \in \mathcal{P}) \land (y = uqv)$.

    Pronounce as $y$ is directly derivable from $x$.

    \item Binary relation $\mathbf{\Rightarrow^*}$ over an grammar $\mathcal{G}$ is defined as reflexive transitive closure of $\Rightarrow$.

    Pronounce as $y$ is derivable from $x$.

    % TODO: use package for BNF
    \item \textbf{Backus-Naur Form (BNF)}\Href{https://en.wikipedia.org/wiki/Backus–Naur_form} is notation to describe the syntax of formal language. A BNF specification is a set of derivation rules, written as:
    $<symbol> ::= \underline{\hspace{4mm}}{}expression\underline{\hspace{4mm}}$, where:

    \begin{terms}
        \item $<symbol>$ is a nonterminal variable that is always enclosed between the pair $<>$.

        \item $::=$ means that the symbol on the left must be replaced with the expression on the right.

        \item $\underline{\hspace{4mm}}{}expression\underline{\hspace{4mm}}$  consists of one or more sequences of either terminal or nonterminal symbols where each sequence is separated by a vertical bar $"|"$ indicating a choice.
    \end{terms}

    %TODO? EBNF.


\end{terms}

\end{document}
