\documentclass[a4paper,12pt]{article}
\usepackage{mypreamble}

%% Page setup
\geometry{
    margin=2cm,
    includehead,
    % includefoot,
    headsep=\baselineskip,
}
\pagestyle{fancy}
\fancyfoot{}
\MakeDoubleHeader% {<l1>}{<l2>}{<r1>}{<r2>}
    {\TextHomeworkRus~\#1}
    {Теория множеств}
    {\TextDiscreteMathRus}
    {\IconFall~Осень 2024}

%% Add custom setup below

\newcommand{\Jaccard}{\mathcal{J}}
\newcommand{\JaccardDist}{d_{\Jaccard}}


\begin{document}

\begin{tasks}
    \item Определите истинность заданных утверждений.
    Считайте, что $a$ и $b$ \--- урэлементы, $a \neq b$.

    \begin{multicols}{3}
    \begin{subtasks}
        \item $a \in \Set{\Set{a}, b}$
        \item $a \in \Set{a, \Set{b}}$
        \item $\Set{a} \in \Set{a, \Set{a}}$
        \item $\Set{a} \subset \Set{a, b}$
        \item $\Set{a} \subseteq \Set{\Set{a}, \Set{b}}$
        \item $\Set{\Set{a}} \subset \Set{\Set{a}, \Set{a, b}}$
        \item $\Set{\Set{a}, b} \subseteq \Set{a, \Set{a, b}, \Set{b}}$
        \item $\begin{multlined}[t]
            \{ a,a \} \union \{ a,a,a \} = \\
            \{ a,a,a,a,a \}
        \end{multlined}$
        \item $\{ a,a \} \union \{ a,a,a \} = \{ a \}$
        \item $\{ a,a \} \intersection \{ a,a,a \} = \{ a \}$
        \item $\{ a,a \} \intersection \{ a,a,a \} = \{ a,a \}$
        \item $\{ a,a,a \} \setminus \{ a,a \} = \{ a \}$
        \item $\emptyset \in \emptyset$
        \item $\emptyset \subseteq \emptyset$
        \item $\emptyset \subset \emptyset$
        \item $\emptyset \in \Set{\emptyset}$
        \item $\emptyset \subseteq \Set{\Set{\emptyset}}$
        \item $\Set{\emptyset, \emptyset} \subset \Set{\emptyset}$
        \item $\Set{\Set{\emptyset}} \subset \Set{\Set{\emptyset}, \Set{\emptyset}}$
        \item $a \in 2^{\Set{a}}$
        \item $2^{\Set{a, \emptyset}} \subset 2^{\Set{a, b, \emptyset}}$
        \item $\Set{a, b} \subseteq 2^{\Set{a, b}}$
        \item $\Set{a, a} \in 2^{\Set{a, a}}$
        \item $\Set{\Set{a}, \emptyset} \subseteq 2^{\Set{a, a}}$
        \item $\Set{a, \Set{a}} \subset 2^{\Set{a, 2^{\Set{a}}}}$
        \item $\Set{\Set{a, \Set{\emptyset}}} \subseteq 2^{\Set{a, 2^{\emptyset}}}$
    \end{subtasks}
    \end{multicols}


    \item Дано множество-универсум\footnote{Здесь под универсумом имеется в виду множество доступных урэлементов. Считайте, что $\overline{A} = \universalset \setminus A$.} $\universalset = \Set{1, 2, \ldots, 10}$ и его подмножества:
    $A = \Set{x \given x \text{ \--- чётное}}$,
    $B = \Set{x \given x \text{ \--- простое\footnotemark}}$,
    $C = \Set{2, 4, 7, 9}$.
    \footnotetext{Считайте, что 1 \href{https://www.google.com/search?q=is 1 a prime number}{не является} простым числом.}%
    Нарисуйте диаграмму Венна для заданных множеств, отметьте на ней все элементы, а затем найдите:

    \begin{multicols}{3}
    \begin{subtasks}
        \item $B \setminus \overline{C}$

        \item $B \symdiff (A \intersection C)$

        \item $\universalset \setminus (\overline{C} \union A)$

        \item $\card{\Set{A \union B \union 2^{\emptyset} \union 2^{\universalset}}}$

        \item $\card{2^{A \setminus C}}$

        \item $(2^{A} \intersection 2^{C}) \setminus 2^{B}$
    \end{subtasks}
    \end{multicols}


    \item Даны следующие множества\footnote{$\square$ \--- самый обыкновенный квадрат, $\Cat$ \--- самый обыкновенный кот.}:
    $A = \Set{1, 2, 4}$,
    $B = \Set{\square, \Cat} \union \emptyset$,
    $C = 2^\emptyset \setminus \Set{\emptyset}$,
    $D = \Set{4, \card{2^{\Set{\emptyset, C}}}}$.
    Внезапно требуется найти:

    \begin{multicols}{3}
    \begin{subtasks}
        \item $A \symdiff D$
        \item $C \times B$
        \item $B \intersection \overline{A}$
        \item $B \times 2^{\Set{C}}$
        \item $D^{\card{C}}$
        \item $\Set{D \intersection \Set{A}} \times (D \union \Set{\card{D}})$
    \end{subtasks}
    \end{multicols}


    \item Мера Жаккара\footnote{\href{https://en.wikipedia.org/wiki/Jaccard_index}{Jaccard index}} $\Jaccard(A, B)$ для двух конечных множеств $A$ и~$B$ определяет степень их похожести и задаётся следующим образом:
    \[
        \Jaccard(A, B) = \frac{\card{A \intersection B}}{\card{A \union B}}
    \]
    При этом $\Jaccard(\emptyset, \emptyset) = 1$.
    Расстояние Жаккара $\JaccardDist(A,B)$ между двумя множествами $A$ и~$B$ определяет степень их различия и задаётся как $d_J(A,B) = 1 - \Jaccard(A,B)$.

    Докажите следующие утверждения для произвольных конечных множеств $A$ и~$B$.

    \begin{subtasks}
        \item $\Jaccard(A,A) = 1$ и $\JaccardDist(A,A) = 0$.
        \item $\Jaccard(A,B) = \Jaccard(B, A)$ и $\JaccardDist(A,B) = \JaccardDist(B, A)$.
        \item $\Jaccard(A,B) = 1$ и $\JaccardDist(A,B) = 0$ тогда и только тогда, когда $A = B$.
        \item $0 \leq \Jaccard(A,B) \leq 1$ и $0 \leq \JaccardDist(A,B) \leq 1$.
        \item Для произвольных множеств $A$, $B$ и~$C$ выполняется \emph{неравенство треугольника}\footnote{Из (a)-(c) и (e) следует, что $\JaccardDist$ является \href{https://en.wikipedia.org/wiki/Metric_space}{\emph{метрикой}}, что крайне интересно и полезно... \textit{для некоторых}.}:
        \[
            \JaccardDist(A,C) \leq \JaccardDist(A,B) + \JaccardDist(B,C)
        \]
    \end{subtasks}


    \item Изобразите на графиках $\Real^2$ следующие множества точек:
    % \footnote{Для всех заданных интервалов (например, $[a;b]$) считайте, что они являются подмножествами $\Real$.}

    \begin{multicols}{2}
    \begin{subtasks}
        \item $\Set{1,2,3} \times (1; 3]$
        \item $[1; 5) \times (1; 4] \setminus \Set{\Pair{2, 3}}$
        \item $[1; 7] \times (1;5] \setminus (1;4] \times (1;3)$
        \item $\Set{\Pair{x,y} \given y \in \Set{1,\dots,5}, x \in [1; 6-y)}$
        \item $\Set{\Pair{x, y} \in [1; 5] \times [1; 4) \given (y \geq x) \lor (x > 4)}$
        \item $\Set{\Pair{x, y} \in (1;5]^2 \given 4(x-2)^2 + 9(y-3)^2 \leq 36}$
    \end{subtasks}
    \end{multicols}


    % \item Пусть $A = \Set{3, \card{B}}$, $B = \Set{1, \card{A}, \card{B}}$.
    % Найдите, чему равны множества $A$ и $B$.


    \item Найдите все множества $A$, $B$ и $C$, которые удовлетворяют следующим условиям:
    \begin{align*}
        A &= \Set{1, \card{B}, \card{C}} \\
        B &= \Set{2, \card{A}, \card{C}} \\
        C &= \Set{1, 2, \card{A}, \card{B}}
    \end{align*}


    \item Нечёткие множества\footnote{\href{https://en.wikipedia.org/wiki/Fuzzy_set}{Fuzzy sets}} \--- обобщение множеств для случаев, когда необходимо описать \textit{вероятностный} или \textit{частичный} характер нахождения элементов во множестве.
    Каждому элементу~$x \in X$ заданного универсума~$X$ сопоставляется \emph{степень принадлежности} $\mu(x) \in [0;1] \subseteq \Real$, задаваемая в виде действительного числа от 0 до~1.
    Нечёткие множества задаются с помощью перечисления элементов вместе со степенями принадлежности, например, $F = \{ a:0.4, b:0.8, c:0.2, d:0.9, e:0.7 \}$ и $R = \{ a:0.6, b:0.9, c:0.4, d:0.1, e:0.5 \}$.

    \begin{subtasks}
        \item Дополнение нечёткого множества~$S$ обозначается~$\overline{S}$ и задаётся как множество, в котором степень принадлежности каждого элемента равна $\mu_{\overline{S}}(x) = 1 - \mu_{S}(x)$.
        Найдите $\overline{F}$ и $\overline{R}$.

        \item Объединение нечётких множеств $S$ и~$T$ обозначается $S \union T$ и задаётся как множество, в котором степень принадлежности каждого элемента есть \emph{максимум} из степеней принадлежности данного элемента в двух исходных множествах $S$ и~$T$.
        Найдите $F \union R$.

        \item Пересечение нечётких множеств $S \intersection T$ задаётся аналогично объединению: $\mu_{S \intersection T}(x) = \min\{\mu_{S}(x), \mu_{T}(x)\}$.
        Найдите $F \intersection R$.

        \item Самостоятельно придумайте определение для разности нечётких множеств $S \setminus T$.
        Найдите $F \setminus R$ и $R \setminus F$.
    \end{subtasks}


    \item Определите счётность или несчётность следующих множеств:

    \begin{subtasks}
        % \item Подмножество счётного множества.
        % \item Надмножество несчётного множества.
        \item Множество рациональных\footnote{Рациональное число можно представить в виде дроби $m / n$, где $m \in \Integer$ \--- целое, а $n \in \Natural$ \--- натуральное.} чисел $\Rational$.
        \item Булеан множества натуральных чисел $\powerset{\Natural}$.
        \item Множество всех функций вида $f : \Natural \to \Natural$.
        \item Объединение \textit{счётного} числа счётных множеств.
        \item Множество действительных корней всех уравнений вида $ax^2 + bx + c = 0$ с целочисленными коэффициентами $a$, $b$ и~$c$.
    \end{subtasks}


    \item Докажите или опровергните следующие утверждения:

    \begin{subtasks}
        \item Если $A \subseteq B$ и $B \subseteq C$, то $A \subseteq C$.
        \item $\card{\powerset{A}} = 2^{\card{A}}$.
        \item $\card{\Complex} = \card{\Real}$, то есть множества комплексных и действительных чисел равномощны.
        \item $\Pair{a,b} = \Pair{c,d} \iff (a = c) \land (b = d)$ при использовании определения упорядоченной пары по Куратовскому: $\Pair{x, y}_K = \Set{\Set{x}, \Set{x, y}}$.
    \end{subtasks}

    % \item \ldots
\end{tasks}

\end{document}
